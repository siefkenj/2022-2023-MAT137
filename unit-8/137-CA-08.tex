\documentclass[14pt]{beamer}

\mode<presentation>{ \usetheme{Madrid}

% To remove the navigation symbols from the bottom of all slides uncomment next line
\setbeamertemplate{navigation symbols}{}
\date{}
\title{}
\author{}

%to get rid of footer entirely uncomment next line
\setbeamertemplate{footline}{}
}

\usepackage{geometry}
\usepackage{multirow}
\usepackage{adjustbox}
\usepackage{multicol}
\setlength{\columnsep}{0.1cm}

\usepackage{tikz}
\usetikzlibrary{shapes, backgrounds}

\usepackage{bbding}
\usepackage{rotating}
\usepackage{xcolor}

%\usepackage{tkz-berge} %cool grid
\usepackage{pgfplots} %pics

\usepackage{graphicx} % Allows including images
\usepackage{
	booktabs
} % Allows the use of \toprule, \midrule and \bottomrule in tables
\usepackage{mathtools}

\newcommand{\R}{\mathbb{R}}
\newcommand{\Z}{\mathbb{Z}}
\newcommand{\N}{\mathbb{N}}
\newcommand{\e}{\varepsilon}

\newcommand{\p}{\pause}

% simple environrment for enumerate, easier to read
\setbeamertemplate{enumerate items}[default]

%%%%%%%%%%%%%%%%%%%%%%

% to use colours easily
\definecolor{verde}{rgb}{0, .7, 0}
\definecolor{rosa}{rgb}{1, 0, 1}
\definecolor{naranja}{rgb}{1, .5, 0.1}
\newcommand{\azul}[1]{{\color{blue} #1}}
\newcommand{\rojo}[1]{{\color{red} #1}}
\newcommand{\verde}[1]{{\color{verde} #1}}
\newcommand{\rosa}[1]{{\color{rosa} #1}}
\newcommand{\naranja}[1]{{\color{naranja} #1}}
\newcommand{\violeta}[1]{{\color{violet} #1}}

% box in red and blue in math and outside of math
\newcommand{\cajar}[1]{\boxed{\mbox{\rojo{ #1}}}}
\newcommand{\majar}[1]{\boxed{\rojo{ #1}}}
\newcommand{\cajab}[1]{\boxed{\mbox{\azul{ #1}}}}
\newcommand{\majab}[1]{\boxed{\azul{ #1}}}

\newcommand{\setsize}[1]{\fontsize{#1}{#1}\selectfont} %allows you to change the font size. The default size of this document is 14. To change the font size of the whole slide, place this at the beginning of the slide. To change the size of only a portion of the text to size 12, you can do the following { \setsize{12} Your text. }.

\setbeamerfont{frametitle}{size=\setsize{15}}
\setbeamerfont{block title}{size=\setsize{14}}

\newcommand{\smallerfont}{\setsize{13}} %place this at the beginning of a slide to set the font size of the entire slide to 13.

%===========================
% Preamble just for this file
%===========================

\newcommand{\erf}{\operatorname{erf}}
\newcommand{\vv}{\vspace{.1cm}}
\setbeamertemplate{enumerate items}{(\Alph{enumi})}

%===================================================
\begin{document}
\begin{frame}
	\frametitle{MAT137 Lecture 42 --- Antiderivatives and Indefinite Integrals}

	\vfill
	{\bf Before next class:}
		\begin{itemize} \normalsize
			\item {\bf Watch videos 8.3, 8.4}
		\end{itemize}
\end{frame}
	\begin{frame}[t]
		\smallerfont
		\frametitle{The most misunderstood antiderivative}
		\begin{enumerate}
			\item Find the \emph{domain} and the derivative of \;
				${\displaystyle F_1(x) = \ln x}$ \vv

			\item Find the \emph{domain} and the derivative of \;
				${\displaystyle F_2(x) = \ln (-x)}$ \vv

			\item Find the \emph{domain} and the derivative of \;
				${\displaystyle F_3(x) = \ln |x|}$ \vv

				\emph{Suggestion:} Break the domain into two pieces. \vv \p

			\item \label{qu:ln} Based on your answers, what is
				${\displaystyle \int \frac{1}{x} \,dx \,}$? \vv \p

			\item Find the \emph{domain} and the derivative of \;
				${\displaystyle F_4(x) = \ln |2x|}$ \vv

				Why doesn't this contradict your answer to \azul{4} ?
		\end{enumerate}
	\end{frame}
	\begin{frame}[t]
		\frametitle{Towards FTC}

		\begin{columns}
			\begin{column}{.7\textwidth}
				\begin{center}
					\includegraphics[scale=.4]{G21}
				\end{center}
			\end{column}

			\begin{column}{.3\textwidth}
				Compute:
				\begin{enumerate}
					\item ${\displaystyle \int_0^1 \! f(t) dt}$

					\item ${\displaystyle \int_0^2 \! f(t) dt}$

					\item ${\displaystyle \int_0^3 \! f(t) dt}$

					\item ${\displaystyle \int_0^4 \! f(t) dt}$

					\item ${\displaystyle \int_0^5 \! f(t) dt}$
				\end{enumerate}
			\end{column}
		\end{columns}
	\end{frame}
	%------------------------------
	\begin{frame}[t]
		\smallerfont
		\frametitle{Towards FTC (continued)}

		\begin{center}
			\includegraphics[width=8cm, height=5cm]{G21}
		\end{center}

		Call ${\displaystyle F(x) = \int_0^x f(t) dt}$. This is a new function.
		\begin{itemize}
			\item Sketch the graph of ${\displaystyle y=F(x)}$.

			\item Using the graph you just sketched, sketch the graph of
				${\displaystyle y=F'(x)}$.
		\end{itemize}
	\end{frame}

	\begin{frame}[t]
		\smallerfont
		\frametitle{Compute these antiderivatives by guess 'n check}

		\begin{enumerate}
			\begin{multicols}{2}
				\item ${\displaystyle \int x^5 dx}$ \item ${\displaystyle \int \left( 3x^8 -18x^5 + 1 \right) dx}$
				\item ${\displaystyle \int \sqrt[3]{x} \; dx}$ \item ${\displaystyle \int \frac{1}{x^{9}} \;dx}$
				\item ${\displaystyle \int \sqrt{x}\left( x^2+ 5 \right) dx}$ \item ${\displaystyle \int \frac{1}{e^{2x}} \; dx}$
				\item ${\displaystyle \int \sin (3x) \; dx}$ \item ${\displaystyle \int \cos (3x+2) \;dx}$
				\item ${\displaystyle \int \sec^2 x \;dx}$ \item ${\displaystyle \int \sec x \tan x \; dx}$
				\item ${\displaystyle \int \frac{1}{x} \; dx}$ \item ${\displaystyle \int \frac{1}{x+3} \; dx}$
			\end{multicols}
		\end{enumerate}
	\end{frame}



\begin{frame}
	\frametitle{MAT137 Lecture 43 --- FTC Part 1}

	\vfill
	{\bf Before next class:}
		\begin{itemize} \normalsize
			\item {\bf Watch videos 8.5, .86, 8.7}
		\end{itemize}
\end{frame}
	\begin{frame}[t]
		\setsize{11}
		\frametitle{Functions defined by integrals}

		Which ones of these are valid ways to define functions?

		\vv

		\begin{multicols}{2}
			\begin{enumerate}
				\item ${\displaystyle F(x) = \int_0^x \frac{t}{1+ t^{8}} \; dt}$
					\vspace{.3cm}

				\item ${\displaystyle F(x) = \int_0^x \frac{x}{1+ x^{8}} \; dx}$
					\vspace{.3cm}

				\item ${\displaystyle F(x) = \int_0^x \frac{x}{1+ t^{8}} \; dt}$
					\vspace{.3cm}

				\item ${\displaystyle F(x) = \int_0^{x^2} \frac{t}{1+t^{8}} \; dt}$
					\vspace{.3cm}

				\item ${\displaystyle F(x) = \int_{\sin x}^{e^x} \frac{t}{1+t^{8}} \; dt}$
					\vspace{.3cm}

				\item ${\displaystyle F(x) = \int_0^3 \frac{t}{1+x^{2}+t^{8}} \; dt}$
					\vspace{.3cm}

				\item ${\displaystyle F(x) = x \int_{\sin x}^{e^x} \frac{t}{1+x^{2}+t^{8}} \; dt}$
					\vspace{.3cm}

				\item ${\displaystyle F(x) = t \int_{\sin x}^{e^x} \frac{t}{1+x^{2}+t^{8}} \; dt}$
					\vspace{.3cm}
			\end{enumerate}
		\end{multicols}
	\end{frame}
	\begin{frame}[t]
		\smallerfont
		\frametitle{True or False?}

		\begin{enumerate}
			\item If $f$ is continuous on the interval $[a,b]$, then
				\[
					\frac{d}{dx}\left( \int_{a}^{b} f(t)dt\right)=f(x).
				\]

				\vspace{4mm}

			\item If $f$ is differentiable, then
				\[
					\frac{d}{dx}\left(\int_{a}^{x} f(t)\,dt \right) \; = \; \int_{a}^{x} f'
					(t) \,dt .
				\]
		\end{enumerate}
	\end{frame}

	\begin{frame}[t]
		\smallerfont
		\frametitle{Examples of FTC-1}

		Compute the derivative of the following functions

		\begin{enumerate}
			\item ${\displaystyle F_1(x) = \int_0^1 e^{-t^2} dt}$. \\
				\vfill

			\item ${\displaystyle F_2(x) = \int_0^x e^{-\sin t} dt}$. \\
				\vfill

			\item ${\displaystyle F_3(x) = \int_1^{x^2} \frac{\sin t}{t^{2}} dt}$. \\
				\vfill

			\item ${\displaystyle F_4(x) = \! \int_x^7 \sin^3 \! \! \left( \sqrt{t} \right) \! dt}$.
				\\
				\vfill

			\item ${\displaystyle F_5(x) = \int_{2x}^{x^2} \frac{1}{1+t^{3}} dt}$. \\
		\end{enumerate}
	\end{frame}

	\begin{frame}[t]
		\frametitle{Creative Guess and Check 1}

		\begin{enumerate}
			\item $\displaystyle \frac{d}{dx}[x\sin x]=$
			\item $\displaystyle \frac{d}{dx}[\cos x]=$

				\bigskip
				\bigskip
			Use the previous answers to compute

			\item $\displaystyle \int x\cos x\,d x=$
		\end{enumerate}
	\end{frame}
	\begin{frame}[t]
		\frametitle{Creative Guess and Check 2}

		\begin{enumerate}
			\item $\displaystyle \frac{d}{dx}[x e^x]=$
			\item $\displaystyle ???$

				\bigskip
				\bigskip
			Use the previous answers to compute

			\item $\displaystyle \int x e^x\,d x=$
		\end{enumerate}
	\end{frame}
	\begin{frame}[t]
		\frametitle{Creative Guess and Check 3}

		\begin{enumerate}
			\item $\displaystyle \frac{d}{dx}[x^2 e^{-x}]=$
			\item $\displaystyle ???$
			\item $\displaystyle ???$

				\bigskip
				\bigskip
			Use the previous answers to compute

			\item $\displaystyle \int x^2 e^{-x}\,d x=$
		\end{enumerate}
	\end{frame}

	\begin{frame}[t]
		\frametitle{Creative Guess and Check 4}

		\begin{enumerate}
			\item $\displaystyle \frac{d}{dx}[x \ln x]=$
			\item $\displaystyle ???$

				\bigskip
				\bigskip
			Use the previous answers to compute

			\item $\displaystyle \int \ln x\,d x=$
		\end{enumerate}
	\end{frame}
	\begin{frame}[t]
		\frametitle{A challenge for guess-and-check ninjas}
		\[
			\int x \, e^{x} \cos x \, dx \; = \; \; ???
		\]
	\end{frame}



\begin{frame}
	\frametitle{MAT137 Lecture 44 --- FTC Part 2}

	\vfill
	{\bf Before next class:}
		\begin{itemize} \normalsize
			\item {\bf Watch videos 9.1, 9.2, 9.3}
		\end{itemize}
\end{frame}

	\begin{frame}[t]
		\frametitle{Compute these definite integrals}

		\begin{enumerate}
			\item ${\displaystyle \int_1^{2} x^3 dx}$
				\vfill

			\item ${\displaystyle \int_0^{1} \left[ e ^x + e^{-x} - \cos (2x) \right] dx}$
				\vfill

			\item ${\displaystyle \int_{1/2}^{1/\sqrt{2}} \frac{4}{\sqrt{1-x^{2}}} dx}$
				\vfill

			\item ${\displaystyle \int_{\pi/4}^{\pi/3} \sec^2 x \; dx}$
				\vfill

			\item ${\displaystyle \int_1^2 \left[ \frac{d}{dx} \left( \frac{\sin^{2} x }{1 + \arctan^{2} x + e^{-x^2}} \right) \right] dx}$
				\vfill
		\end{enumerate}
	\end{frame}
	\begin{frame}[t]
		\frametitle{Find the error}

		\[
			\int_{-1}^{1}\frac{1}{x^{4}}dx \; = \; \left. \frac{-1}{3x^{3}}\right\vert_{-1}
			^{1}= \frac{-2}{3}
		\]

		However, $x^{4}$ is always positive, so the integral should be positive.
	\end{frame}


	\begin{frame}[t]
		\frametitle{Areas}

		Calculate the area of the bounded region...
		\vspace{.2cm}
		\begin{enumerate}
			\item ... between the $x$-axis and ${\displaystyle y=4x-x^2}$.
				\vspace{.2cm}

			\item ... between $y=\cos x$, the $x$-axis, from $x=0$ to $x=\pi$.
				\vspace{.2cm}

			\item ... between ${\displaystyle y=x^2+3}$ and ${\displaystyle y=3x+1}$.
				\vspace{.2cm}

			\item ... between $y=1$, the $y$-axis, and $y=\ln(x+1)$.
		\end{enumerate}
	\end{frame}
	\begin{frame}[t]
		\smallerfont
		\frametitle{More True or False}

		Let $f$ and $g$ be differentiable functions with domain ${\displaystyle \R}$.
		\\ Assume that ${\displaystyle f'(x) = g(x)}$ for all $x$. \\ Which of the
		following statements must be true?

		\begin{enumerate}
			\item ${\displaystyle f(x) = \int_0^x g(t) dt}$.

			\item If ${\displaystyle f(0)=0}$, then
				${\displaystyle f(x) = \int_0^x g(t) dt}$.

			\item If ${\displaystyle g(0)=0}$, then
				${\displaystyle f(x) = \int_0^x g(t) dt}$.

			\item There exists $C \in \mathbb{R}$ such that ${\displaystyle f(x) = C + \int_0^x g(t) dt}$.

			\item There exists $C\in \mathbb{R}$ such that ${\displaystyle f(x) = C + \int_1^x g(t) dt}$.
		\end{enumerate}
	\end{frame}






%	%===================================================
%
%	%----------------------------------------------------------------------------------------
%	%	Antiderivatives
%	%----------------------------------------------------------------------------------------
%	%-----------------------------
%	\begin{frame}[t]
%		\frametitle{Initial Value Problem}
%
%		Find a function $f$ such that
%		\begin{itemize}
%			\item For every ${\displaystyle x \in \R}$,
%				${\displaystyle f''(x) = \sin x + x^2}$,
%
%			\item ${\displaystyle f'(0) = 5}$,
%
%			\item ${\displaystyle f(0) = 7}$.
%		\end{itemize}
%	\end{frame}
%	%-----------------------------
%	%-----------------------------
%	%-----------------------------
%	\begin{frame}[t]
%		\frametitle{Integration by parts 1}
%
%		\begin{enumerate}
%			\item ${\displaystyle \frac{d}{dx} \left[ x \sin x \right] = }$ \vv \vv
%
%			\item ${\displaystyle \frac{d}{dx} \left[ \cos x \right] = }$ \vv \vv
%		\end{enumerate}
%		\vv \vv Use the previous answers to calculate \vv \vv
%		\begin{enumerate}
%			\addtocounter{enumi}{2}
%
%			\item ${\displaystyle \int x \cos x \, dx = }$
%		\end{enumerate}
%	\end{frame}
%	%-----------------------------
%	\begin{frame}[t]
%		\frametitle{Integration by parts 2}
%
%		\begin{enumerate}
%			\item ${\displaystyle \frac{d}{dx} \left[ x e^x \right] = }$
%				\vspace{.5cm}
%
%			\item ${\displaystyle ??? }$
%				\vspace{.5cm}
%
%			\item ${\displaystyle \int x e^x \, dx = }$
%		\end{enumerate}
%	\end{frame}
%	%-----------------------------
%	\begin{frame}[t]
%		\frametitle{Integration by parts 3}
%
%		\begin{enumerate}
%			\item ${\displaystyle ??? }$
%				\vspace{.3cm}
%
%			\item ${\displaystyle ??? }$
%				\vspace{.3cm}
%
%			\item ${\displaystyle \int x e^{-x} \, dx = }$
%		\end{enumerate}
%	\end{frame}
%	%-----------------------------
%	\begin{frame}[t]
%		\frametitle{Integration by parts 4}
%
%		\begin{enumerate}
%			\item ${\displaystyle \frac{d}{dx} \left[ x^2 e^x \right] = }$ \vv \vv
%
%			\item ${\displaystyle \frac{d}{dx} \left[ x e^x \right] = }$
%				\vspace{.5cm}
%
%			\item ${\displaystyle ??? }$
%				\vspace{.5cm}
%
%			\item ${\displaystyle \int x^2 e^x \, dx = }$
%		\end{enumerate}
%	\end{frame}
%	%-----------------------------
%	\begin{frame}[t]
%		\frametitle{Trig-exp antiderivatives}
%
%		\begin{enumerate}
%			\item ${\displaystyle \frac{d}{dx} \left[ e^x \sin x \right]=}$ \vv \vv
%
%			\item ${\displaystyle \frac{d}{dx} \left[ e^x \cos x \right]=}$
%		\end{enumerate}
%		\vspace{.5cm}
%		Use the previous answers to calculate:
%		\vspace{.3cm}
%		\begin{enumerate}
%			\addtocounter{enumi}{2}
%
%			\item ${\displaystyle \int e^x \sin x \, dx = }$ \vv \vv
%
%			\item ${\displaystyle \int e^x \cos x \, dx = }$
%		\end{enumerate}
%	\end{frame}
%	%-----------------------------
%	%-----------------------------
%	%----------------------------------------------------------------------------------------
%	%	Functions defined as integrals
%	%----------------------------------------------------------------------------------------
%	%-----------------------------
%	%------------------------------
%	%-----------------------------
%	%----------------------------------------------------------------------------------------
%	%	FTC - 1
%	%----------------------------------------------------------------------------------------
%	%------------------------------
%	\begin{frame}[t]
%		\frametitle{Filling the tank}
%
%		A tank is being filled with water. At time $t$ water flows into the tank at a
%		rate of
%		\[
%			A \, e^{-bt}\arctan (ct)
%		\]
%		litres per second, where $A$, $b$, and $c$ are constants. The amount of water
%		in the tank at time $t=0s$ is $V_{0}$. Write an expression for the amount of
%		water $V$ in the tank at time $t$.
%	\end{frame}
%
%	%-----------------------------
%	%-----------------------------
%	%-----------------------------
%	\begin{frame}[t]
%		\setsize{11}
%		\frametitle{True, False, or Shrug?}
%
%		We want to find a function $H$ with domain $\R$ such that ${\displaystyle H(1) = -2}$
%		and such that ${\displaystyle H'(x) = e^{\sin x}}$ for all $x$. Decide
%		whether each of the following statements is true, false, or we do not have
%		enough information to decide.
%
%		\begin{enumerate}
%			\item The function ${\displaystyle \quad H(x) = \int_0^x e^{\sin t} dt \quad}$
%				is a solution.
%
%			\item The function ${\displaystyle \quad H(x) = \int_2^x e^{\sin t} dt \quad}$
%				is a solution.
%
%			\item $\forall C \in \R$, the function
%				${\displaystyle \quad H(x) = \int_0^x e^{\sin t} dt + C \quad}$ is a
%				solution.
%
%			\item $\exists C \in \R$ s.t.\ the function
%				${\displaystyle \quad H(x) = \int_0^x e^{\sin t} dt + C \quad }$ is a
%				solution.
%
%			\item The function ${\displaystyle \quad H(x) = \int_1^x e^{\sin t} dt -2 \quad}$
%				is a solution.
%
%			\item There is more than one solution.
%		\end{enumerate}
%	\end{frame}
%	%------------------------------
%	%------------------------------
%	\begin{frame}[t]
%		\frametitle{A generalized version of FTC-1}
%
%		Let $f$, $u$, $v$ be differentiable functions with domain $\R$. Let us call
%		\[
%			F(x) = \int_{u(x)}^{v(x)}f(t) dt
%		\]
%		Find a formula for
%		\[
%			F'(x)
%		\]
%		in terms of $f$, $u$, $v$, $f'$, $u'$, $v'$.
%	\end{frame}
%
%	%------------------------------
%	\begin{frame}[t]
%		\frametitle{An integral equation}
%
%		Assume $f$ is a continuous function that satisfies, for every $x \in \R$:
%		\[
%			\int_{0}^{x} e^{t} f(t) = \frac{\sin x}{x^{2}+1}
%		\]
%
%		Find an explicit expression for $f(x)$.
%	\end{frame}
%	%-----------------------------
%	%----------------------------------------------------------------------------------------
%	%	FTC - 2
%	%----------------------------------------------------------------------------------------
%	%------------------------------
%	%------------------------------
%	%-----------------------------
%	%-----------------------------
%	\begin{frame}[t]
%		\frametitle{Minimizing area}
%
%		For each $a >0$ consider the function
%		\[
%			f_{a}(x) = 1 + a -ax^{2}
%		\]
%
%		Find the value of $a$ that minimizes the area of the region bounded by the graph
%		of $f_{a}$ and the $x$-axis. \hfill
%		\href{https://www.desmos.com/calculator/x7vkfcerdp}{\beamergotobutton{desmos}}
%	\end{frame}
%	%-----------------------------
%	\begin{frame}[t]
%		\frametitle{Symmetry}
%
%		Calculate the value of these integrals \emph{without computing any
%		antiderivative}.
%
%		\begin{enumerate}
%			\begin{multicols}{3}
%				\item ${\displaystyle \int_{-2}^{2} \sin x^3 dx }$ \item ${\displaystyle \int_0^{\pi} \cos^2 x \, dx}$
%				\item ${\displaystyle \int_{-1}^{1} \arccos x \, dx}$
%			\end{multicols}
%		\end{enumerate}
%
%		\emph{Hint:} Sketch the graphs (use desmos) and use symmetry to compute the integral.
%		\\ Once you guess the symmetry of the graph, try to write it algebraically.
%		\hfill
%		\href{https://www.desmos.com/calculator/ncysdsu3yv}{\beamergotobutton{1}}
%		\href{https://www.desmos.com/calculator/fwjs5zoury}{\beamergotobutton{2}}
%		\href{https://www.desmos.com/calculator/tjakgza6vf}{\beamergotobutton{3}}
%	\end{frame}
%
%	%-----------------------------
%	\begin{frame}[t]
%		\smallerfont
%		\frametitle{Average Velocity}
%
%		You are traveling. \\ Your position at time $t$ is $s(t)$. \\ Your velocity
%		at time $t$ is $v(t)$. \\ The function $v$ is continuous on an interval $[a,b
%		]$.
%
%		Which of the following represent your average velocity on $[a,b]$?
%
%		\
%
%		\begin{enumerate}
%			\item ${\displaystyle \frac{s(b) - s(a)}{b-a}}$
%
%			\item ${\displaystyle \frac{1}{b-a} \int_a^b v(t) dt}$
%
%				\
%
%
%			\item $v(c)$ for at least one $c$ between $a$ and $b$
%		\end{enumerate}
%	\end{frame}
%	%------------------------------
%	\begin{frame}[t]
%		\smallerfont
%		\frametitle{The Mean Value Theorem for integrals is back}
%
%		Prove the following theorem.
%
%		\begin{block}{\smallerfont Theorem}
%			Let $a < b$. Let $f$ be a continuous function on $[a,b]$. \\ There exists $c
%			\in [a,b]$ such that
%			\vspace{-.5cm}
%			\[
%				f(c) \; = \; \frac{1}{b-a}\int_{a}^{b} f(t) dt
%			\]
%		\end{block}
%
%		\p
%
%		\emph{Hint:} Use MVT for the function ${\displaystyle F(x) = \int_a^x f(t) dt}$.
%	\end{frame}
%	%------------------------------
%	%-----------------------------
\end{document}
%-----------------------------
%-----------------------------
