\documentclass[14pt]{beamer}

\mode<presentation>{ \usetheme{Madrid}

% To remove the navigation symbols from the bottom of all slides uncomment next line
\setbeamertemplate{navigation symbols}{}
\date{}
\title{}
\author{}

%to get rid of footer entirely uncomment next line
\setbeamertemplate{footline}{}
}

\usepackage{geometry}
\usepackage{multirow}
\usepackage{adjustbox}
\usepackage{multicol}
\setlength{\columnsep}{0.1cm}

\usepackage{tikz}
\usetikzlibrary{shapes, backgrounds}

\usepackage{bbding}
\usepackage{rotating}
\usepackage{xcolor}

%\usepackage{tkz-berge} %cool grid
\usepackage{pgfplots} %pics

\usepackage{graphicx} % Allows including images
\usepackage{
	booktabs
} % Allows the use of \toprule, \midrule and \bottomrule in tables
\usepackage{mathtools}

\newcommand{\R}{\mathbb{R}}
\newcommand{\Z}{\mathbb{Z}}
\newcommand{\N}{\mathbb{N}}
\newcommand{\e}{\varepsilon}

\newcommand{\p}{\pause}

% simple environrment for enumerate, easier to read
\setbeamertemplate{enumerate items}[default]

%%%%%%%%%%%%%%%%%%%%%%

% to use colours easily
\definecolor{verde}{rgb}{0, .7, 0}
\definecolor{rosa}{rgb}{1, 0, 1}
\definecolor{naranja}{rgb}{1, .5, 0.1}
\newcommand{\azul}[1]{{\color{blue} #1}}
\newcommand{\rojo}[1]{{\color{red} #1}}
\newcommand{\verde}[1]{{\color{verde} #1}}
\newcommand{\rosa}[1]{{\color{rosa} #1}}
\newcommand{\naranja}[1]{{\color{naranja} #1}}
\newcommand{\violeta}[1]{{\color{violet} #1}}

% box in red and blue in math and outside of math
\newcommand{\cajar}[1]{\boxed{\mbox{\rojo{ #1}}}}
\newcommand{\majar}[1]{\boxed{\rojo{ #1}}}
\newcommand{\cajab}[1]{\boxed{\mbox{\azul{ #1}}}}
\newcommand{\majab}[1]{\boxed{\azul{ #1}}}

\newcommand{\setsize}[1]{\fontsize{#1}{#1}\selectfont} %allows you to change the font size. The default size of this document is 14. To change the font size of the whole slide, place this at the beginning of the slide. To change the size of only a portion of the text to size 12, you can do the following { \setsize{12} Your text. }.

\setbeamerfont{frametitle}{size=\setsize{15}}
\setbeamerfont{block title}{size=\setsize{14}}

\newcommand{\smallerfont}{\setsize{13}} %place this at the beginning of a slide to set the font size of the entire slide to 13.
\setbeamertemplate{enumerate items}{(\Alph{enumi})}

%===========================
% Preamble just for this file
%===========================

%===================================================
\begin{document}
\begin{frame}
	\frametitle{MAT137 Lecture 55 --- Improper Integrals}

	\vfill
	{\bf Before next class:}
		\begin{itemize} \normalsize
			\item {\bf Watch videos 12.7, 12.8 }
		\end{itemize}
\end{frame}
	\begin{frame}[t]
		\frametitle{Recall the definitions}

		\begin{enumerate}
			\item {\bf Type-1 improper integrals.} Let $f$ be a bounded, continuous
				function on ${\displaystyle [c, \infty)}$. How do we define the improper
				integral
				\[
					\int_{c}^{\infty}f(x) dx \, ?
				\]

				\vfill

			\item {\bf Type-2 improper integrals.} Let $f$ be a continuous function on
				${\displaystyle (a,b]}$, possibly with $x=a$ as a vertical asymptote. How
				do we define the improper integral
				\[
					\int_{a}^{b}f(x) dx \, ?
				\]

				\vfill
		\end{enumerate}
	\end{frame}
	\begin{frame}[t]
		\frametitle{Computation}

		Calculate, using the definition of improper integral
		\[
			\int_{1}^{\infty}\frac{1}{x^{2}+x}dx
		\]
		\
 \emph{Hint:}
		${\displaystyle \frac{1}{x^{2}+x} = \frac{(x+1) - (x)}{x(x+1)}}$
	\end{frame}
	%------------------------------
	\begin{frame}[t]
		\frametitle{The most important improper integrals}

		Use the definition of improper integral to determine for which values of
		${\displaystyle p \in \R}$ each of the following improper integrals
		converges.

		\begin{enumerate}
			\item ${\displaystyle \int_1^{\infty} \frac{1}{x^{p}} \, dx }$
				\vfill

			\item ${\displaystyle \int_0^1 \frac{1}{x^{p}} \, dx }$
				\vfill

			\item ${\displaystyle \int_0^{\infty} \frac{1}{x^{p}} \, dx }$
				\vfill
		\end{enumerate}
	\end{frame}

	\begin{frame}[t]
		\smallerfont
		\frametitle{Positive functions}

		\begin{itemize}
			\item Let $f$ be continuous on $[a, \infty)$. Let
				${\displaystyle  A = \int_a^{\infty} f(x) dx }$

				Then $A$ may be ${\displaystyle  \begin{cases}\mbox{ convergent (a number) } \\ \mbox{ divergent } \begin{cases}\mbox{ to } \infty \\ \mbox{ to } - \infty \\ \mbox{ ``oscillating"}\end{cases}\end{cases} }$

				\ \p

			\item Assume ${\displaystyle \forall x \geq a, f(x) \geq 0}$.
				\vspace{.2cm}

				\boxed{\mbox{Which of the four options are still possible? }}

				\ \p

			\item Assume ${\displaystyle \exists M \geq a, \mbox{ s.t. } \forall x \geq M, f(x) \geq 0}$.
				\vspace{.2cm}

				\boxed{\mbox{Which of the four options are still possible? }}
		\end{itemize}
	\end{frame}

\begin{frame}
	\frametitle{MAT137 Lecture 56 --- The Basic Comparison Test}

	\vfill
	{\bf Before next class:}
		\begin{itemize} \normalsize
			\item {\bf Watch videos 12.9, 12.10 }
		\end{itemize}
\end{frame}

	\begin{frame}[t]
		\frametitle{Quick review}

		For which values of ${\displaystyle p \in \R}$ is each of the following
		improper integrals convergent?

		\begin{enumerate}
			\item ${\displaystyle \int_1^{\infty} \frac{1}{x^{p}} \, dx }$

				\vfill

			\item ${\displaystyle \int_0^1 \frac{1}{x^{p}} \, dx }$

				\vfill

			\item ${\displaystyle \int_0^{\infty} \frac{1}{x^{p}} \, dx }$
		\end{enumerate}
	\end{frame}

	\begin{frame}[t]
		\smallerfont
		\frametitle{A simple BCT application}

		We want to determine whether
		${\displaystyle \int_1^{\infty} \frac{1}{x+e^{x}} \; dx}$ \\ is convergent
		or divergent.

		\
 We can try at least two comparisons:

		\begin{enumerate}
			\item Compare ${\displaystyle \frac{1}{x}}$ and ${\displaystyle \frac{1}{x+ e^{x}}}$.

			\item Compare ${\displaystyle \frac{1}{e^{x}}}$ and ${\displaystyle \frac{1}{x+ e^{x}}}$.
		\end{enumerate}

		\
 Try both. What can you conclude from each one of them?
	\end{frame}

	\begin{frame}[t]
		\setsize{11}
		\frametitle{True or False - Comparisons}

		Let $a \in \R$. \\ Let $f$ and $g$ be continuous functions on $[a, \infty)$.
		\\ Assume that ${\displaystyle \boxed{\forall x \geq a, \quad 0 \leq \azul{f(x)} \leq \rojo{g(x)}}}$.
		\\ What can we conclude?

		\begin{enumerate}
			\item IF ${\displaystyle \azul{\int_a^{\infty} f(x) dx}}$ is convergent,
				\, THEN ${\displaystyle \rojo{\int_a^{\infty} g(x) dx}}$ is convergent.

			\item IF ${\displaystyle \azul{\int_a^{\infty} f(x) dx} = \infty}$, \,
				THEN ${\displaystyle \rojo{\int_a^{\infty} g(x) dx} = \infty}$.

			\item IF ${\displaystyle \rojo{\int_a^{\infty} g(x) dx}}$ is convergent,
				\, THEN ${\displaystyle \azul{\int_a^{\infty} f(x) dx}}$ is convergent.

			\item IF ${\displaystyle \rojo{\int_a^{\infty} g(x) dx} = \infty}$, \,
				THEN ${\displaystyle \azul{\int_a^{\infty} f(x) dx} = \infty}$.
		\end{enumerate}
	\end{frame}

	\begin{frame}[t]
		\setsize{11}
		\frametitle{True or False - Comparisons II}

		Let $a \in \R$. \\ Let $f$ and $g$ be continuous functions on $[a, \infty)$.
		\\ Assume that ${\displaystyle \boxed{\forall x \geq a, \quad \azul{f(x)} \leq \rojo{g(x)}}}$.
		\\ What can we conclude?

		\begin{enumerate}
			\item IF ${\displaystyle \azul{\int_a^{\infty} f(x) dx}}$ is convergent,
				\, THEN ${\displaystyle \rojo{\int_a^{\infty} g(x) dx}}$ is convergent.

			\item IF ${\displaystyle \azul{\int_a^{\infty} f(x) dx} = \infty}$, \,
				THEN ${\displaystyle \rojo{\int_a^{\infty} g(x) dx} = \infty}$.

			\item IF ${\displaystyle \rojo{\int_a^{\infty} g(x) dx}}$ is convergent,
				\, THEN ${\displaystyle \azul{\int_a^{\infty} f(x) dx}}$ is convergent.

			\item IF ${\displaystyle \rojo{\int_a^{\infty} g(x) dx} = \infty}$, \,
				THEN ${\displaystyle \azul{\int_a^{\infty} f(x) dx} = \infty}$.
		\end{enumerate}
	\end{frame}
	%------------------------------
	\begin{frame}[t]
		\setsize{11}
		\frametitle{True or False - Comparisons III}

		Let $a \in \R$. \\ Let $f$ and $g$ be continuous functions on $[a, \infty)$.
		\\ Assume that ${\displaystyle \boxed{\exists M \geq a \; \mbox{ s.t. } \; \forall x \geq M, \quad 0 \leq \azul{f(x)} \leq \rojo{g(x)}}}$.
		\\ What can we conclude?

		\begin{enumerate}
			\item IF ${\displaystyle \azul{\int_a^{\infty} f(x) dx}}$ is convergent,
				\, THEN ${\displaystyle \rojo{\int_a^{\infty} g(x) dx}}$ is convergent.

			\item IF ${\displaystyle \azul{\int_a^{\infty} f(x) dx} = \infty}$, \,
				THEN ${\displaystyle \rojo{\int_a^{\infty} g(x) dx} = \infty}$.

			\item IF ${\displaystyle \rojo{\int_a^{\infty} g(x) dx}}$ is convergent,
				\, THEN ${\displaystyle \azul{\int_a^{\infty} f(x) dx}}$ is convergent.

			\item IF ${\displaystyle \rojo{\int_a^{\infty} g(x) dx} = \infty}$, \,
				THEN ${\displaystyle \azul{\int_a^{\infty} f(x) dx} = \infty}$.
		\end{enumerate}
	\end{frame}

	\begin{frame}[t]
		\frametitle{BCT calculations}

		Use BCT to determine whether each of the following is convergent or
		divergent

		\begin{enumerate}
			\begin{multicols}{2}
				\item ${\displaystyle \int_1^{\infty} \frac{1 + \cos^{2} x}{x^{2/3}} \, dx}$
				\vspace{.3cm}
				\item ${\displaystyle \int_1^{\infty} \frac{1 + \cos^{2} x}{x^{4/3}} \, dx}$
				\vspace{.3cm}
				\item ${\displaystyle \int_0^{\infty} \frac{\arctan x^{2} }{1 + e^{x}} \, dx}$
				\vspace{.3cm}
				\p \item ${\displaystyle \int_0^{\infty} e^{-x^2} dx}$
				\vspace{.3cm}
				\item ${\displaystyle \int_2^{\infty} \frac{(\ln x)^{10}}{x^{2}} \, dx}$
				\vspace{.3cm}
			\end{multicols}
		\end{enumerate}
	\end{frame}






\begin{frame}
	\frametitle{MAT137 Lecture 57 --- The Limit Comparison Test}

	\vfill
	{\bf Before next class:}
		\begin{itemize} \normalsize
			\item {\bf Watch videos 13,2, 13.3, 13.4 }
		\end{itemize}
\end{frame}


	\begin{frame}[t]
		\frametitle{Rapid questions: convergent or divergent?}

		\begin{enumerate}
			\begin{multicols}{3}
				\item ${\displaystyle \int_1^{\infty} \frac{1}{x^{2}} \, dx }$
				\vspace{.4cm}
				\item ${\displaystyle \int_1^{\infty} \frac{1}{\sqrt{x}} \, dx }$
				\vspace{.4cm}
				\item ${\displaystyle \int_1^{\infty} \frac{1}{x} \, dx }$
				\vspace{.4cm}
				\item ${\displaystyle \int_0^{1} \frac{1}{x^{2}} \, dx }$
				\vspace{.3cm}
				\item ${\displaystyle \int_0^{1} \frac{1}{\sqrt{x}} \, dx }$
				\vspace{.3cm}
				\item ${\displaystyle \int_0^{1} \frac{1}{x} \, dx }$
				\vspace{.3cm}
				\item ${\displaystyle \int_1^{\infty} \frac{3}{x^{2}} \, dx }$
				\vspace{.3cm}
				\item ${\displaystyle \int_1^{\infty} \frac{1}{x^{2}+3} \, dx }$
				\vspace{.3cm}
				\item ${\displaystyle \int_1^{\infty} \! \! \! \left( \frac{1}{x^{2}} + 3 \right) \! dx }$
				\vspace{.3cm}
			\end{multicols}
		\end{enumerate}
	\end{frame}


	\begin{frame}[t]
		\frametitle{A ``simple" integral}

		What is ${\displaystyle \int_{-1}^{1} \frac{1}{x} \, dx}$ \; ? \p

		\begin{enumerate}
			\vfill

			\item ${\displaystyle \int_{-1}^{1} \frac{1}{x} \, dx \; = \; \left( \ln |x| \right) \Big\vert_{-1}^{1} \; = \; \ln|1| - \ln|-1| = 0}$
				\vfill

			\item ${\displaystyle \int_{-1}^{1} \frac{1}{x} \, dx = 0}$ \; because ${\displaystyle f(x) = \frac{1}{x}}$
				is an odd function.
				\vfill

			\item ${\displaystyle \int_{-1}^{1} \frac{1}{x} \, dx}$ is divergent.
				\vfill
		\end{enumerate}
	\end{frame}

	\begin{frame}[t]
		\smallerfont
		\frametitle{Slow questions: convergent or divergent?}

		\begin{enumerate}
			\begin{multicols}{2}
				\item ${\displaystyle \int_1^{\infty} \! \! \frac{x^{3} + 2x + 7}{x^{5} + 11x^{4}+ 1} \, dx }$
				\vspace{.4cm}
				\item ${\displaystyle \int_1^{\infty} \! \! \frac{1}{\sqrt{x^{2}+x+1}} \, dx}$
				\vspace{.4cm}
				\item ${\displaystyle \int_0^{1} \frac{3 \cos x}{x + \sqrt{x}} \, dx }$
				\vspace{.4cm}
				\item ${\displaystyle \int_0^1 \sqrt{\cot x} \, dx}$
				\vspace{.4cm}
				\item ${\displaystyle \int_0^1 \frac{\sin x}{x^{3/2}} \, dx }$
				\vspace{.4cm}
				\item ${\displaystyle \int_0^{\infty} e^{-x^2} dx}$
				\vspace{.4cm}
				\item ${\displaystyle \int_2^{\infty} \frac{(\ln x)^{10}}{x^{2}} \, dx}$
				\vspace{.4cm}
			\end{multicols}
		\end{enumerate}
	\end{frame}

	\begin{frame}[t]
		\frametitle{What is wrong with this computation?}

		\[
			\begin{aligned}
				\int_{-1}^{1}\frac{1}{x}\, dx \; & = \; \lim_{\e \to 0^+}\left[ \int_{-1}^{-\e}\frac{1}{x}\, dx \; + \; \int_{\e}^{1}\frac{1}{x}\, dx \, \right]                                            \\
				\;                               & = \; \lim_{\e \to 0^+}\left[ \left. \phantom{\frac{1}{1}}\ln|x| \right\vert_{-1}^{-\e}\; + \left. \phantom{\frac{1}{1}}\ln|x| \right\vert_{\e}^{1}\; \right] \\
				\;                               & = \; \lim_{\e \to 0^+}\left[ \, \ln|-\e| - \ln |\e| \, \right] \phantom{\int}                                                                            \\
				\;                               & = \; \lim_{\e \to 0^+}\left[ \, 0 \, \right] \; = \; 0 \phantom{\int}
			\end{aligned}
		\]
	\end{frame}











%	%===================================================
%	%----------------------------------------------------------------------------------------
%	%	Definition of improper integral
%	%----------------------------------------------------------------------------------------
%	%------------------------------
%
%	%------------------------------
%	%------------------------------
%	%------------------------------
%	\begin{frame}[t]
%		\frametitle{Examples}
%
%		\begin{enumerate}
%			\item Let $f$ be continuous on $[a, \infty)$. Let
%				${\displaystyle  A = \int_a^{\infty} f(x) dx }$
%
%				Then $A$ may be ${\displaystyle  \begin{cases}\mbox{ convergent (a number) } \\ \mbox{ divergent } \begin{cases}\mbox{ to } \infty \\ \mbox{ to } - \infty \\ \mbox{ ``oscillating"}\end{cases}\end{cases} }$
%
%				\
% Give one example of each of the four results.
%				\vfill
%				\p
%
%			\item Now do the same thing for ``type 2" improper integrals.
%		\end{enumerate}
%		\vfill
%	\end{frame}
%	%------------------------------
%	%------------------------------
%	%------------------------------
%	%------------------------------
%	\begin{frame}[t]
%		\smallerfont
%		\frametitle{Probability}
%
%		\smallerfont
%		\vspace{-2mm}
%
%		A nonnegative function $f$ defined on $(-\infty,\infty)$ is called a {\bf probability density function }
%		if
%		\vspace{-2mm}
%		\[
%			\int_{-\infty}^{\infty}f(x)\, dx=1.
%		\]
%		The \emph{mean} of a probability density function is defined as
%		\vspace{-2mm}
%		\[
%			\mu=\int_{-\infty}^{\infty}x \, f(x)\, dx.
%		\]
%		Let
%		${\displaystyle f(x) = \begin{cases}Ce^{-kx}&\mbox{ if }x\geq 0 \\ 0&\mbox{ if }x <0\end{cases} }$
%		\begin{enumerate}
%			\item For $k>0$, find a constant $C$ such that the function $f$ is a
%				probability density function.
%
%			\item Calculate the mean $\mu$.
%		\end{enumerate}
%	\end{frame}
%	%------------------------------
%	\begin{frame}[t]
%		\smallerfont
%		\frametitle{Collection of antiderivatives}
%
%		Let ${\displaystyle f}$ be a positive, continuous function with domain $\R$.
%		\\ We know two ways to describe a collection of antiderivatives:
%		\begin{enumerate}
%			\item ${\displaystyle G(x) + C}$ for $C \in \R$, where $G$ is any one
%				antiderivative.
%
%			\item The collection of functions ${\displaystyle F_a}$ for ${\displaystyle a \in \R}$,
%				where
%				\[
%					F_{a}(x) = \int_{a}^{x} f(t) dt
%				\]
%		\end{enumerate}
%
%		\p These two collections are not always the same. Why not? Are they the same
%		for some functions $f$? When are they the same?
%
%		\vspace{2.5cm}
%		\hfill \emph{Hint:} \quad
%		\href{https://tinyurl.com/137antiderivatives}{\beamergotobutton{https://tinyurl.com/137antiderivatives}}
%	\end{frame}
%	%------------------------------
%	%----------------------------------------------------------------------------------------
%	%	Comparison tests
%	%----------------------------------------------------------------------------------------
%	%------------------------------
%	%------------------------------
%	%------------------------------
%	%------------------------------
%	\begin{frame}[t]
%		\smallerfont
%		\frametitle{What can you conclude?}
%
%		Let $a \in \R$. Let $f$ be a continuous, {\bf positive} function on
%		$[a, \infty)$. \\ In each of the following cases, what can you conclude
%		about ${\displaystyle \int_a^{\infty} f(x) dx}$? \; Is it convergent,
%		divergent, or we do not know?
%
%		\begin{enumerate}
%			\item ${\displaystyle \forall b \geq a, \; \exists M \in \R \; \mbox{ s.t. } \; \quad \int_a^b f(x)\,dx \; \leq \; M}$.
%
%			\item ${\displaystyle \exists M \in \R \; \mbox{ s.t. } \; \forall b \geq a, \; \quad \int_a^b f(x)\, dx \; \leq \; M}$.
%				\vspace{.3cm}
%
%			\item ${\displaystyle \exists M >0 \; \mbox{ s.t. } \; \quad \forall x \geq a, \; f(x) \leq M}$.
%				\vspace{.3cm}
%
%			\item ${\displaystyle \exists M >0 \; \mbox{ s.t. } \; \quad \forall x \geq a, \; f(x) \geq M}$.
%		\end{enumerate}
%	\end{frame}
%
%	%------------------------------
%	%------------------------------
%	%------------------------------
%	%------------------------------
%	\begin{frame}[t]
%		\frametitle{A harder calculation}
%
%		For which values of $a>0$ is the integral
%		\[
%			\int_{0}^{\infty}\frac{\arctan x}{x^{a}}\, dx
%		\]
%		convergent?
%	\end{frame}
%	%------------------------------
%	\begin{frame}[t]
%		\setsize{12}
%		\frametitle{A variation on LCT}
%
%		This is the theorem you have learned:
%		\begin{block}{\smallerfont Theorem (Limit-Comparison Test)}
%			Let ${\displaystyle a \in \R}$. Let $f$ and $g$ be positive, continuous functions
%			on ${\displaystyle [a, \infty)}$.
%			\begin{itemize}
%				\item IF the limit \; ${\displaystyle  L = \lim_{x \to \infty} \frac{f(x)}{g(x)} }$
%					\; exists and \; ${\displaystyle L>0}$
%
%				\item THEN ${\displaystyle  \int_a^{\infty} \! \! f(x) dx}$ \; and \; ${\displaystyle  \int_a^{\infty} \! \! g(x) dx }$
%					\\ are both convergent or both divergent.
%			\end{itemize}
%		\end{block}
%
%		\p
%		\vspace{.2cm}
%		{\bf What if we change the hypotheses to $L=0$?}
%		\begin{enumerate}
%			\item Write down the new theorem (different conclusion).
%
%			\item Prove it.
%		\end{enumerate}
%
%		\p
%		\vspace{.2cm}
%		\emph{Hint:} If ${\displaystyle \lim_{x \to \infty} \frac{f(x)}{g(x)}=0 }$,
%		what is larger $f(x)$ or $g(x)$?
%	\end{frame}
%	%------------------------------
%	\begin{frame}[t]
%		\setsize{12}
%		\frametitle{A variation on LCT - 2}
%
%		This is the theorem you have learned:
%		\begin{block}{\smallerfont Theorem (Limit-Comparison Test)}
%			Let ${\displaystyle a \in \R}$. Let $f$ and $g$ be positive, continuous functions
%			on ${\displaystyle [a, \infty)}$.
%			\begin{itemize}
%				\item IF the limit \; ${\displaystyle  L = \lim_{x \to \infty} \frac{f(x)}{g(x)} }$
%					\; exists and \; ${\displaystyle L>0}$
%
%				\item THEN ${\displaystyle  \int_a^{\infty} \! \! f(x) dx}$ \; and \; ${\displaystyle  \int_a^{\infty} \! \! g(x) dx }$
%					\\ are both convergent or both divergent.
%			\end{itemize}
%		\end{block}
%
%		\vspace{.2cm}
%		{\bf What if we change the hypotheses to $L=\infty$?}
%		\begin{enumerate}
%			\item Write down the new theorem (different conclusion).
%
%			\item Prove it.
%		\end{enumerate}
%	\end{frame}
%	%------------------------------
%	\begin{frame}[t]
%		\setsize{12}
%		\frametitle{Absolute Convergence}
%		\vspace{-.3cm}
%
%		\begin{block}{\setsize{12} Definition}
%			The integral ${\displaystyle \int_a^\infty f(x)\, dx}$ is called \textbf{absolutely
%			convergent} when ${\displaystyle \int_a^\infty |f(x)|\, dx}$ converges.
%		\end{block}
%
%		Prove that
%		\begin{itemize}
%			\item IF an improper integral is absolutely convergent
%
%			\item THEN it is convergent
%		\end{itemize}
%
%		\vspace{.5cm}
%
%		\textit{Hint:} Consider the functions
%		\[
%			f_{+}(x) =
%			\begin{cases}
%				f(x) & \mbox{ if } f(x) \geq 0 \\
%				0    & \mbox{ if } f(x) \leq 0
%			\end{cases}
%			\quad \quad f_{-}(x) =
%			\begin{cases}
%				0      & \mbox{ if }f(x) \geq 0 \\
%				|f(x)| & \mbox{ if }f(x) \leq 0
%			\end{cases}
%		\]
%		Write $f(x)$ and $|f(x)|$ in terms of $f_{+}(x)$ and $f_{-}(x)$. Use BCT.
%	\end{frame}
%	%------------------------------
%	\begin{frame}[t]
%		\setsize{12}
%		\frametitle{Dirichlet integral}
%
%		Let ${\displaystyle f(x) = \begin{cases}\frac{\sin x}{x} & \mbox{ if } x \neq 0 \\ 1 & \mbox{ if } x = 0\end{cases} }$
%		\vspace{.1cm}
%
%		\begin{enumerate}
%			\item Is \; ${\displaystyle \int_0^1 \! f(x) \, dx}$ \; an inproper integral?
%
%			\item Show that \; ${\displaystyle \int_1^{\infty} \frac{\cos x}{x^{2}} \, dx}$
%				\;is absolutely convergent.
%				\vspace{.1cm}
%
%				\emph{Hint:} Use BCT.
%
%			\item The same argument is inconclusive for ${\displaystyle \int_1^{\infty} \! \! f(x) \, dx}$.
%				Why?
%
%			\item Show that \; ${\displaystyle \int_1^{\infty} \! \! f(x) dx}$ \; is convergent
%				\vspace{.1cm}
%
%				\emph{Hint:} Use the definition of improper integral, not comparison
%				tests. Use integration by parts with $u=\frac{1}{x}$ and $dv= \sin x \,dx$.
%				\vspace{.1cm}
%		\end{enumerate}
%
%		{\setsize{10} \emph{Note:} It is possible to prove that \; ${\displaystyle \int_1^{\infty} \! \! \frac{\sin x}{x} \, dx}$ is not absolutely convergent. }
%	\end{frame}

	%-----------------------------
\end{document}
%-----------------------------
%-----------------------------
